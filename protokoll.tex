\documentclass{school}
\usepackage{minted}
\usemintedstyle{rainbow_dash}

\title{Wordpress}
\subject{INSY - Webentwicklung}
\author{Markus Reichl}

\newacronym{ac-cms}{CMS}{Content Management System}

\begin{document}
\maketitle
\thispagestyle{fancy}	% Makes the first page fancy too

\tableofcontents

\section{Aufgabe}
Es soll ein Webauftritt für ein Diplomprojekt mittels Wordpress, inklusive Präsentation des Projektes und mehreren Postings uber den Projektfortschritt gestaltet werden.
Relevant hier weniger der Inhalt sondern vielmehr, sondern die Umsetzung. Speziell:
\begin{outline}
\1 Das Design baut auf Bootstrap (oder auch einem anderen CSS Framework)
\1 Es soll ein eigenes Wordpress-Theme, zur Darstellung der Inhalte aus den Posts auf Basis einer Vorlage gestaltet werden.
\1 Es soll ein eigenes Wordpress-Plugin entwickelt und hinzugefügt werden.
\1 Inhalte sollen mit speziellen Posts und eigenen Feldern eingebaut werden.
\end{outline}

\newpage
\section{Installation}
Wordpress\cite{wp} kann über die Seite\cite{wp-dl} des \gls{ac-cms} heruntergeladen und nach der beiliegenden Anleitung\cite{wp-help} installiert werden.
\\\\
In diesem Fall wurde Wordpress, zusammen mit einer MySQL\cite{mysql} Datenbank und den zugehörigen Abhängigkeiten über den Virtualisierungsdienst Docker\cite{docker} installiert.

Die Konfiguration, sowie die Steuerung wird dabei über Docker Compose\cite{docker-compose} vorgenommen, wobei die \texttt{docker-compose.yml} wie folgt definiert wurde.

\inputminted{yaml}{docker-compose.yml}

Der Container kann dabei über \texttt{docker-compose up -d} gestartet und mittels \texttt{docker-compose down} heruntergefahren werden.

\section{}

% Basic Figure
% \begin{figure}[h]
%	 \centering
% 	 \includegraphics[height=4cm]{image.jpg}
% 	 \caption{Caption}
% \end{figure

\newpage
% Basic bibiography
\begin{thebibliography}{9}
\bibitem{wp} Wordpress. Blog Tool, Publishing Platform, and CMS \\ https://wordpress.org
\bibitem{wp-dl} Wordpress. Download WordPress \\ https://wordpress.org/latest.zip
\bibitem{wp-help} Wordpress. Installing WordPress \\ https://codex.wordpress.org/Installing\_WordPress
\bibitem{docker} Docker. Build, Ship and Run Any App, Anywhere \\ https://www.docker.com
\bibitem{docker-compose} Docker. Docker Compose | Docker Documentation \\ https://docs.docker.com/compose/
\bibitem{mysql} MySQL \\ https://www.mysql.com/
\end{thebibliography}

% List of figures
% \listoffigures
\end{document}